% Options for packages loaded elsewhere
\PassOptionsToPackage{unicode}{hyperref}
\PassOptionsToPackage{hyphens}{url}
%
\documentclass[
]{article}
\usepackage{amsmath,amssymb}
\usepackage{lmodern}
\usepackage{iftex}
\ifPDFTeX
  \usepackage[T1]{fontenc}
  \usepackage[utf8]{inputenc}
  \usepackage{textcomp} % provide euro and other symbols
\else % if luatex or xetex
  \usepackage{unicode-math}
  \defaultfontfeatures{Scale=MatchLowercase}
  \defaultfontfeatures[\rmfamily]{Ligatures=TeX,Scale=1}
\fi
% Use upquote if available, for straight quotes in verbatim environments
\IfFileExists{upquote.sty}{\usepackage{upquote}}{}
\IfFileExists{microtype.sty}{% use microtype if available
  \usepackage[]{microtype}
  \UseMicrotypeSet[protrusion]{basicmath} % disable protrusion for tt fonts
}{}
\makeatletter
\@ifundefined{KOMAClassName}{% if non-KOMA class
  \IfFileExists{parskip.sty}{%
    \usepackage{parskip}
  }{% else
    \setlength{\parindent}{0pt}
    \setlength{\parskip}{6pt plus 2pt minus 1pt}}
}{% if KOMA class
  \KOMAoptions{parskip=half}}
\makeatother
\usepackage{xcolor}
\IfFileExists{xurl.sty}{\usepackage{xurl}}{} % add URL line breaks if available
\IfFileExists{bookmark.sty}{\usepackage{bookmark}}{\usepackage{hyperref}}
\hypersetup{
  hidelinks,
  pdfcreator={LaTeX via pandoc}}
\urlstyle{same} % disable monospaced font for URLs
\usepackage{color}
\usepackage{fancyvrb}
\newcommand{\VerbBar}{|}
\newcommand{\VERB}{\Verb[commandchars=\\\{\}]}
\DefineVerbatimEnvironment{Highlighting}{Verbatim}{commandchars=\\\{\}}
% Add ',fontsize=\small' for more characters per line
\newenvironment{Shaded}{}{}
\newcommand{\AlertTok}[1]{\textcolor[rgb]{1.00,0.00,0.00}{\textbf{#1}}}
\newcommand{\AnnotationTok}[1]{\textcolor[rgb]{0.38,0.63,0.69}{\textbf{\textit{#1}}}}
\newcommand{\AttributeTok}[1]{\textcolor[rgb]{0.49,0.56,0.16}{#1}}
\newcommand{\BaseNTok}[1]{\textcolor[rgb]{0.25,0.63,0.44}{#1}}
\newcommand{\BuiltInTok}[1]{#1}
\newcommand{\CharTok}[1]{\textcolor[rgb]{0.25,0.44,0.63}{#1}}
\newcommand{\CommentTok}[1]{\textcolor[rgb]{0.38,0.63,0.69}{\textit{#1}}}
\newcommand{\CommentVarTok}[1]{\textcolor[rgb]{0.38,0.63,0.69}{\textbf{\textit{#1}}}}
\newcommand{\ConstantTok}[1]{\textcolor[rgb]{0.53,0.00,0.00}{#1}}
\newcommand{\ControlFlowTok}[1]{\textcolor[rgb]{0.00,0.44,0.13}{\textbf{#1}}}
\newcommand{\DataTypeTok}[1]{\textcolor[rgb]{0.56,0.13,0.00}{#1}}
\newcommand{\DecValTok}[1]{\textcolor[rgb]{0.25,0.63,0.44}{#1}}
\newcommand{\DocumentationTok}[1]{\textcolor[rgb]{0.73,0.13,0.13}{\textit{#1}}}
\newcommand{\ErrorTok}[1]{\textcolor[rgb]{1.00,0.00,0.00}{\textbf{#1}}}
\newcommand{\ExtensionTok}[1]{#1}
\newcommand{\FloatTok}[1]{\textcolor[rgb]{0.25,0.63,0.44}{#1}}
\newcommand{\FunctionTok}[1]{\textcolor[rgb]{0.02,0.16,0.49}{#1}}
\newcommand{\ImportTok}[1]{#1}
\newcommand{\InformationTok}[1]{\textcolor[rgb]{0.38,0.63,0.69}{\textbf{\textit{#1}}}}
\newcommand{\KeywordTok}[1]{\textcolor[rgb]{0.00,0.44,0.13}{\textbf{#1}}}
\newcommand{\NormalTok}[1]{#1}
\newcommand{\OperatorTok}[1]{\textcolor[rgb]{0.40,0.40,0.40}{#1}}
\newcommand{\OtherTok}[1]{\textcolor[rgb]{0.00,0.44,0.13}{#1}}
\newcommand{\PreprocessorTok}[1]{\textcolor[rgb]{0.74,0.48,0.00}{#1}}
\newcommand{\RegionMarkerTok}[1]{#1}
\newcommand{\SpecialCharTok}[1]{\textcolor[rgb]{0.25,0.44,0.63}{#1}}
\newcommand{\SpecialStringTok}[1]{\textcolor[rgb]{0.73,0.40,0.53}{#1}}
\newcommand{\StringTok}[1]{\textcolor[rgb]{0.25,0.44,0.63}{#1}}
\newcommand{\VariableTok}[1]{\textcolor[rgb]{0.10,0.09,0.49}{#1}}
\newcommand{\VerbatimStringTok}[1]{\textcolor[rgb]{0.25,0.44,0.63}{#1}}
\newcommand{\WarningTok}[1]{\textcolor[rgb]{0.38,0.63,0.69}{\textbf{\textit{#1}}}}
\setlength{\emergencystretch}{3em} % prevent overfull lines
\providecommand{\tightlist}{%
  \setlength{\itemsep}{0pt}\setlength{\parskip}{0pt}}
\setcounter{secnumdepth}{-\maxdimen} % remove section numbering
\ifLuaTeX
  \usepackage{selnolig}  % disable illegal ligatures
\fi

\author{}
\date{}

\begin{document}

\hypertarget{iv-analysis}{%
\section{IV. Analysis}\label{iv-analysis}}

\hypertarget{41-identify-boundarycontrollerentity-and-tool-classes}{%
\subsection{4.1 Identify Boundary,Controller,Entity, and Tool
classes}\label{41-identify-boundarycontrollerentity-and-tool-classes}}

We separated the source codes of the project into 4 parts:
Boundary,Controller,Entity, and Tool classes.

\hypertarget{411-boundary-class}{%
\subsubsection{4.1.1 Boundary Class}\label{411-boundary-class}}

The boundary class shoulders the responsibility for presenting our
Graphic User Interface , listening to their actions, and interacting
with them. The boundary class is at the top of our software architecture
and only interacts with the controller classes, making our software
loose coupling. In our system, the boundary classes include:

\begin{Shaded}
\begin{Highlighting}[]
\NormalTok{ AdditionalServiceView.java}
\NormalTok{ CheckinView.java}
\NormalTok{ ChooseFlightView.java}
\NormalTok{ ChooseMealView.java}
\NormalTok{ ChooseSeatView.java}
\NormalTok{ ConfirmationView.java}
\NormalTok{ CreditcardView.java}
\NormalTok{ IDNoCheckinView.java}
\NormalTok{ WelcomeView.java}
\end{Highlighting}
\end{Shaded}

\hypertarget{412-controller-class}{%
\subsubsection{4.1.2 Controller Class}\label{412-controller-class}}

We use the controller class to link the Boundary class to the Entity
class, which processes data requests sent by the Boundary class and
linking to the corresponding JSON-data file based on the content of the
request. After getting the data from the JSON-data file in the project,
it parses the return packet and returns the specific data to the
Boundary class. In our system, the controller classes include:

\begin{Shaded}
\begin{Highlighting}[]
\NormalTok{AdditionalServiceController.java}
\NormalTok{CheckinController.java}
\NormalTok{ChooseFlightController.java}
\NormalTok{ChooseMealController.java}
\NormalTok{ChooseSeatController.java}
\NormalTok{ConfirmationController.java}
\NormalTok{Controller.java}
\NormalTok{CreditCardController.java}
\NormalTok{IDNoCheckinController.java}
\NormalTok{WelcomeController.java}
\end{Highlighting}
\end{Shaded}

\hypertarget{413-entity-class}{%
\subsubsection{4.1.3 Entity Class}\label{413-entity-class}}

The entity class in our project manages the data administration. It
consists of the basic data structure of the passengers data and all the
get and set methods to modify and search. In our system, entity class
includes:

\begin{Shaded}
\begin{Highlighting}[]
\NormalTok{BookingInformation.java}
\end{Highlighting}
\end{Shaded}

The get and set methods include:

\begin{Shaded}
\begin{Highlighting}[]
     \FunctionTok{getFirstName}\OperatorTok{()}
	 \FunctionTok{setFirstName}\OperatorTok{(}\BuiltInTok{String}\OperatorTok{)}
	 \FunctionTok{getLastName}\OperatorTok{()}
	 \FunctionTok{setLastName}\OperatorTok{(}\BuiltInTok{String}\OperatorTok{)}
	 \FunctionTok{getFlightNumber}\OperatorTok{()}
	 \FunctionTok{setFlightNumber}\OperatorTok{(}\BuiltInTok{String}\OperatorTok{)}
	 \FunctionTok{getBoardingTime}\OperatorTok{()}
	 \FunctionTok{setBoardingTime}\OperatorTok{(}\BuiltInTok{String}\OperatorTok{)}
	 \FunctionTok{getBoardingGate}\OperatorTok{()}
	 \FunctionTok{setBoardingGate}\OperatorTok{(}\BuiltInTok{String}\OperatorTok{)}
	 \FunctionTok{getSeat}\OperatorTok{()}
	 \FunctionTok{setSeat}\OperatorTok{(}\BuiltInTok{String}\OperatorTok{)}
	 \FunctionTok{getPrimaryFood}\OperatorTok{()}
	 \FunctionTok{setPrimaryFood}\OperatorTok{(}\BuiltInTok{String}\OperatorTok{)}
	 \FunctionTok{getCredFirst}\OperatorTok{()}
	 \FunctionTok{setCredFirst}\OperatorTok{(}\BuiltInTok{String}\OperatorTok{)}
	 \FunctionTok{getCredSecond}\OperatorTok{()}
	 \FunctionTok{setCredSecond}\OperatorTok{(}\BuiltInTok{String}\OperatorTok{)}
	 \FunctionTok{getCredNumber}\OperatorTok{()}
	 \FunctionTok{setCredNumber}\OperatorTok{(}\BuiltInTok{String}\OperatorTok{)}
	 \FunctionTok{getSecurCode}\OperatorTok{()}
	 \FunctionTok{setSecurCode}\OperatorTok{(}\BuiltInTok{String}\OperatorTok{)}
	 \FunctionTok{setExtraServiceFee}\OperatorTok{(}\DataTypeTok{int}\OperatorTok{)}
	 \FunctionTok{getExtraServiceFee}\OperatorTok{()}
	 \FunctionTok{setExtraService}\OperatorTok{(}\BuiltInTok{String}\OperatorTok{)}
	 \FunctionTok{getStartWhere}\OperatorTok{()}
	 \FunctionTok{getDestWhere}\OperatorTok{()}
	 \FunctionTok{getDuringTime}\OperatorTok{()}
	 \FunctionTok{getBookingNo}\OperatorTok{()}
	 \FunctionTok{getExtraService}\OperatorTok{()}
	 \FunctionTok{setStartWhere}\OperatorTok{(}\BuiltInTok{String}\OperatorTok{)}
	 \FunctionTok{setDestWhere}\OperatorTok{(}\BuiltInTok{String}\OperatorTok{)}
	 \FunctionTok{setDuringTime}\OperatorTok{(}\BuiltInTok{String}\OperatorTok{)}
	 \FunctionTok{setBookingNo}\OperatorTok{(}\DataTypeTok{int}\OperatorTok{)}
	 \FunctionTok{getIdNo}\OperatorTok{()}
	 \FunctionTok{setIdNo}\OperatorTok{(}\BuiltInTok{String}\OperatorTok{)}
	 \FunctionTok{getFilePath}\OperatorTok{()}
	 \FunctionTok{setFilePath}\OperatorTok{(}\BuiltInTok{String}\OperatorTok{)}
	 \FunctionTok{getSeatHelpNumber}\OperatorTok{()}
	 \FunctionTok{setSeatHelpNumber}\OperatorTok{(}\DataTypeTok{int}\OperatorTok{)}
\end{Highlighting}
\end{Shaded}

\hypertarget{414-tool-class}{%
\subsubsection{4.1.4 Tool class}\label{414-tool-class}}

The Tool class handles the file reading and file writing for the JSON
files which store the data as a database. \texttt{FileReaderWriter}
class handles the Java file reading and writing, while \texttt{JsonTool}
handles the conversion of data into JSON to store. The Tool classes
include:

\begin{Shaded}
\begin{Highlighting}[]
\NormalTok{ FileReaderWriter.java}
\NormalTok{ FileReaderWriterTest.java}
\NormalTok{ JsonTool.java}
\end{Highlighting}
\end{Shaded}

\hypertarget{42-conceptual-class-diagram}{%
\subsection{4.2 Conceptual Class
Diagram}\label{42-conceptual-class-diagram}}

Having analyzed the relationships and associations between classes, we
drew the conceptual dia-gram. (Click here to view the whole Conceptual
Class Diagram)

\hypertarget{43-reusability}{%
\subsection{4.3 Reusability}\label{43-reusability}}

To make our code re-usable, we store our data in an appropriate data
structure and design our code following the design process. For example,
we create JSON file \texttt{checkinData}to store all the check in
data,\texttt{flightData} file to store all the plane information, and
\texttt{passengerData} file to store the passengers data. Thus, the code
can easily adapt to changes, the system can add new categories and new
membership is up a notch by just modifying the JSON file.\\
Besides, the methodology we used to store data into JSON file is just a
new way to store the data aside from sql-database technique.

\end{document}
